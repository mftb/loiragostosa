\documentclass[a4paper,10pt]{article}

\usepackage[brazilian]{babel}
\usepackage[utf8]{inputenc}
\usepackage[T1]{fontenc}
\usepackage{amsmath}
\usepackage{algorithmic}
\usepackage{amssymb}

\title{Sistemas de recomendacoes}
\author{Alex Grilo}

\begin{document}
\maketitle


\textbf{Introducao}

O objetivo deste relatorio e descrever os assuntos estudados durante a 
disciplina de MC032 no segundo semestre de 2010. O principal interesse
deste trabalho foi pesquisar na literatura sobre
o problema de sistemas de recomendacoes e diferentes abordagens para resolve-lo. 
Na secao 2 e descrito em linhas gerais o que e um sistema de recomendacoes. 
Nas secoes 3 e 4 sao mostradas as
principais ideias encontradas nos trabalhos  ,
respectivamente. 
Na secao 5 encontra-se a conclusao do estudo e ideias para trabalhos futuros.

\textbf{Sistemas de recomendacoes}

Um sistema de recomendacoes tem como objetivo recomendar a usuarios
produtos que o agradem. Para isso, o sistema de recomedacoes se baseia
inicialmente em respostas ja conhecidas de outros usuarios e alguma
informacao sobre o usuario que deseja obter uma recomendacao. 
O conjunto de informacao inicial que o sistema de recomendacoes ja
possui e utilizado para comparar com a informacao obtida do usuario
em questao para tentar deduzir quais outros produtos interessam
ao usuario. 

\textbf{Competitive Recommendation Systems}

\textit{Introducao}

A ideia basica desta abordagem e reduzir o problema de sistema de recomendacoes
no problema de reconstrucao de matrizes a partir de informacoes parciais da mesma. 
Foram utilizadas estrategias de reconstrucao de matrizes
baseadas na tecnica de SVD, encontrando boas aproximacoes
para a matriz original 

\textit{Divisao em tipos}
Assume-se que os usuarios podem ser divididos em grupos com interesses semelhantes,
classificando-os em tipos. 
Consideraremos que $k$ seja o numero de tipos, onde $k$ e
uma constante que nao depende do numero de usuarios.

\textit{Qualidade dos algoritmos}

Um bom algoritmo de recomendacoes deve dar boas recomendacoes utilizando o minimo de 
informacao possivel. Sao estabelecidas 2 metricas para medir a qualidade de um algoritmo
de recomendacoes.

 Um algoritmo e $c-competitivo$ $para$ $amostragem$ se usa somente $ck$
linhas e colunas da matriz para recomendar mais produtos.

Um algoritmo e $f-competitivo$ $para$ $utilidade$
se faz boas recomendacoes a $fm$ usuarios, onde $m$ e o numero de usuarios. 

\textit{Notacao}

Nesta secao, a seguinte notacao sera utilizada:

\begin{itemize}
\item[A] matriz de recomendacao original
\item[Aij] valor da utilidade do produto j para o usuario i
\item[A(i)] i-esima linha da matriz A, que corresponde ao
vetor de utilidades do i-esimo usuario.
\item[A(i)] i-esima coluna a matriz A, que corresponde a 
utilidade de um produto para todos os usuarios.
\item[Ak] Melhor aproximacao com posto k de A, obtida atraves da tecnica de SVD.
\item[air] valor da utilidade do r-esimo produto com maior utilidade
para o usuario i 
\end{itemize}

\textit{Reconstrucao de matrizes}

\textit{Escolha das linhas}

A escolha de um numero constante de linhas segue a ideia de que sao necessarios
alguns usuarios respondendo um questionario sobre todos os produtos para que os
outros usuarios sejam classificados a partir das respostas destes. Na pratica, empresas
pagam pessoas para o preenchimento desses formularios.

Portanto a partir da escolha das $ak$ linhas, podemos formar uma matriz $V$ de tipos efetivos.

\textit{Escolha das colunas}

A ideia de escolher um numero constante de colunas acontece pois todo usuario
que deseja obter uma recomendacao deve responder inicialmente um pequeno 
questionario, dando alguma informacao sua para que o algoritmo de recomendacao
\textbf{Ideia da prova}

Primeiramente separa-se os casos em que o usuario faz parte do comite e quando
o usuario nao faz 
Para o caso em que o usuario faz parte do comite, ele tera que fazer $K$ provas.
Para o caso em que o usuario nao esta no comite, e calculada a esperanca do numero 
de provas que ele tera que fazer. Como a ordem dos produtos
recomendados e aleatoria, o numero de provas que um usuario tera que fazer
e uma variavel aleatoria geometrica. Logo, basta fazer a analise da esperanca
para essa variavel aleatoria \cite{MG}.
O resultado do teorema aparece com a esperanca do numero de provas para
cada caso.

\textit{Comite}
A tecnica de criar um comite do algoritmo proposto possui serias
desvantagens. 
A primeira delas e que a complexidade de recomendacao individual
e $\Omega(n)$, ja que os usuarios do comite tem que provar todos os
algoritmos. 
Um segundo ponto contra esta tecnica e a existencia de usuarios desonestos,
que, quando escolhidos para o comite, atrapalham o funcionamento do 
algoritmo de recomendacoes \cite{Boney96}.

\textbf{Conclusao e trabalhos futuros}

Atraves deste estudo, verificou-se a existencia de solucoes com abordagens
totalmente diferentes para o problema de sistemas de recomendacoes. Mas o mais
importante foi o aprendizado das tecnicas mais gerais para resolucao de problemas
como por exemplo SVD. 
Potenciais trabalhos futuros sao a implementacao e comparacao dos algoritmos na pratica
ou alteracao neles para que ao inves de dispor de informacoes de colunas inteiras, os
algoritmos funcionem com posicoes esparsas na matriz. 


\begin{thebibliography}
    \bibitem{Boney96} Boney, L., Tewfik, A.H., and Hamdy, K.N., Digital 
        Watermarks for Audio Signals, Proceedings of the Third IEEE
        International Conference on Multimedia, pp. 473-480, June 1996.
    \bibitem{MG} Goossens, M., Mittelbach, F., Samarin, A LaTeX Companion, Addison-Wesley, Reading, MA, 1994.
    \bibitem{HK} Kopka, H., Daly P.W., A Guide to LaTeX, Addison-Wesley, Reading, MA, 1999.
    \bibitem{Pan} Pan, D., A Tutorial on MPEG/Audio Compression, IEEE Multimedia, Vol.2, pp.60-74, Summer 1998.
\end{thebibliography}


\end{document}


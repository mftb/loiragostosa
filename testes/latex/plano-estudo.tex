\documentclass[a4paper,10pt]{article}

\usepackage[brazilian]{babel}
\usepackage[utf8]{inputenc}
\usepackage[T1]{fontenc}
\usepackage{amsmath}
\usepackage{amssymb}

\title{Plano de trabalho - Mestrado}
\author{Alex Bredariol Grilo}

\begin{document}
\maketitle

\textbf{Computacao quantica}

Deseja-se neste projeto realizar um estudo em largura nos temas em computacao quantica e
escolher alguns pontos especificos deste estudo e fazer um aprofundamento, buscando
estudar os avancos mais recentes da area e tentar contribuir para a
evolucao do conhecimento nesta area. Ha o objetivo do resultado do estudo se tornar uma introducao ao tema de computacao
quantica na lingua portuguesa

Topicos que serao cobertos no estudo em largura da area em computacao quantica:
\begin{itemize}
\item Diferencas entre bits classicos, probabilisticos e quanticos
\item Circuitos quanticos
\item Algoritmos quanticos tradicionais
 \begin{itemize}
 \item Algoritmo de Deutsch e Deutsch-Josza
 \item Algoritmo de Shor
 \item Algoritmo de Grover
 \end{itemize}
\end{itemize}

Possiveis topicos para estudo em profundidade
\begin{itemize}
\item Linguagens de programacao quanticas
\item Quantum walks
\item Maquinas de estado quanticas
\item Maquinas de Turing quanticas
\end{itemize}

\end{document}
